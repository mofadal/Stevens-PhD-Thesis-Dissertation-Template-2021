% NYU PhD thesis format. Original template created by José Koiller 2007--2008.
%% Updated by Anshul Vikram Pandey with new design guidelines. 2017--2018.
%%% Modified by Abdullah Khanfor for Stevens Institute of Technology PhD thesis format design guidelines 2019--2020. Link to the old template: https://github.com/khanfor/stevens-phd-thesis-dissertation-template
%%% Updated by Mofadal Alymani for Stevens Institute of Technology PhD thesis format design guidelines 2020--2021.

%% Use the first of the following lines during production to
%% easily spot "overfull boxes" in the output. Use the second
%% line for the final version.
% \documentclass[12pt,draft,letterpaper]{report}
% \documentclass[12pt,letterpaper]{report}
\documentclass[12pt]{report}
\usepackage{siunitx}
\usepackage{enumitem}
\usepackage{url}
\usepackage[breaklinks]{hyperref}
\def\UrlBreaks{\do\/\do-}
%% Replace the title, name, advisor name, graduation date and dedication below with
%% your own. Graduation months must be January, May or September.
\newcommand{\thesistitle}{Thesis Title}
\newcommand{\thesisauthor}{First Last}
\newcommand{\thesisadvisor}{Advisor}
\newcommand{\thesisyear}{2021}
\newcommand{\thesisname}{First Last}
\newcommand{\thesischairadvisor}{Dr. Advisor}    % this name prints on the title page as chairman and the abstract page as advisor
\newcommand{\committeenameA}{Dr. Member}
\newcommand{\committeenameB}{Dr. Member}
\newcommand{\committeenameC}{Dr. Member}
\newcommand{\thesisdepartment}{Department}
\newcommand{\thesisdate}{April 26, 2021}
\newcommand{\thesistype}{dissertation}
\newcommand{\thesisdegree}{Doctor of Philosophy}
\newcommand{\graddate}{\the\year} % like 2020, 2019, no month or day should be written
\newcommand{\thesissigline}[1]{%
  \leftline{\hbox to 2.5in{}\hrulefill}
  \endgraf
  \vspace*{-18pt}
  \leftline{\hbox to 2.53in{}{#1}}}

%% If you do not want a dedication, scroll down and comment out
%% the appropriate lines in this file.
\newcommand{\thesisdedication}{To all the Ph.D. pursuing brave souls}

%% The following makes chapters and sections, but not subsections,
%% appear in the TOC (table of contents). Increase to 2 or 3 to
%% make subsections or subsubsections appear, respectively. It seems
%% to be usual to use the "1" setting, however.
\setcounter{tocdepth}{1}

%% Sectional units up to subsubsections are numbered. To number
%% subsections, but not subsubsections, decrease this counter to 2.
\setcounter{secnumdepth}{3}

% Setting a gap between page number and text block

%% This inputs your auxiliary file with \usepackage's and \newcommand's:
%% It is assumed that that file is called "definitions.tex".
%%
%% Place here your \usepackage's. Some recommended packages are already included.
%%

% Graphics:
\usepackage[final]{graphicx}
%\usepackage{graphicx} % use this line instead of the above to suppress graphics in draft copies
%\usepackage{graphpap} % \defines the \graphpaper command

% Indent first line of each section:
%\usepackage{indentfirst}

% Good AMS stuff:
\usepackage{amsthm} % facilities for theorem-like environments
\usepackage[tbtags]{amsmath} % a lot of good stuff!

% Fonts and symbols:
\usepackage{amsfonts}
\usepackage{amssymb}

\usepackage{xspace}

% Algorithm figures
\usepackage{algorithm}
\usepackage[noend]{algpseudocode}

\usepackage{microtype}
\usepackage{subfigure}
\usepackage{color}
\usepackage{todonotes}
\usepackage{url}
\newfloat{algorithm}{t}{lop}

\usepackage{blindtext}

%% Controls spacing between lines (\doublespacing, \onehalfspacing, etc.):
\usepackage[utf8x]{inputenc}
\usepackage{fancyhdr}

% This package to change the font if the document. This font is optional as you preference. You can comment it to use the CMU font
%\usepackage{helvet}
%\renewcommand{\familydefault}{\sfdefault}

%% \usepackage{amsmath}
%% \usepackage{amssymb}
\usepackage{lipsum}
% \newfloat{algorithm}{t}{lop}

% Packages for setting the length and width of document
\usepackage{setspace}

% Package for sideway images and figures
\usepackage{rotating}
\usepackage{pdflscape}

\usepackage{wrapfig}
\usepackage{lscape}
\usepackage{epstopdf}

\usepackage{caption}

% Formatting tools:
%\usepackage{relsize} % relative font size selection, provides commands \textsmalle, \textlarger
%\usepackage{xspace} % gentle spacing in macros, such as \newcommand{\acims}{\textsc{acim}s\xspace}

% Page formatting utility:
%\usepackage{geometry}
\usepackage{multirow}

% For MATLAB code
\usepackage{listings}
\usepackage[framed,numbered,autolinebreaks,useliterate]{mcode}

% For citations
\usepackage[numbers,sort]{natbib}
\usepackage[nottoc]{tocbibind}

\usepackage[all,cmtip]{xy}

% Change the color of the hyperlinks and titles from blue to black
\usepackage{hyperref}
\hypersetup{
    colorlinks = false,
    linkbordercolor = {white},
    linkcolor=black,
    filecolor=black,
    urlcolor=black,
    citecolor=black
}

%%
%% Place here your \newcommand's and \renewcommand's. Some examples already included.
%%
%\newcommand{\acims}{\textsc{acim}s\xspace}
\newcommand{\Mspace}        {{\mathbb M}}
\newcommand{\Rspace}        {{\mathbb R}}
\newcommand{\Cspace}        {{\mathbb C}}

\newcommand{\Mo}        {{\hat M}}
\newcommand{\Ms}        {{\tilde M}}
\newcommand{\Do}          {{\hat D}}
\newcommand{\Ds}        {{\tilde D}}
\newcommand{\doo}          {{\hat d}}
\newcommand{\dss}        {{\tilde d}}
\newcommand{\w}        {{\mathbf w}}

% general
\newcommand{\ie}{i.e.}
\newcommand{\eg}{e.g.}
\newcommand{\reffig}[1]{{Figure~\ref{#1}}}
\newcommand{\refchap}[1]{{Chapter~\ref{#1}}}
\newcommand{\refsec}[1]{{Section~\ref{#1}}}
\newcommand{\reftab}[1]{{Table~\ref{#1}}}
\newcommand{\refapp}[1]{{Appendix~\ref{#1}}}
\newcommand{\refeq}[1]{{Equation~\ref{#1}}}
\newcommand{\refalg}[1]{{Algorithm~\ref{#1}}}
\newcommand{\myparagraph}[1]{\noindent \textbf{#1}}
\newcommand{\highlight}[1]{{\color{black}#1}}

%%
%% Place here your \newtheorem's:
%%

%% Some examples commented out below. Create your own or use these...
%%%%%%%%%\swapnumbers % this makes the numbers appear before the statement name.
%\theoremstyle{plain}
%\newtheorem{thm}{Theorem}[chapter]
%\newtheorem{prop}[thm]{Proposition}
%\newtheorem{lemma}[thm]{Lemma}
%\newtheorem{cor}[thm]{Corollary}

%\theoremstyle{definition}
%\newtheorem{define}{Definition}[chapter]

%\theoremstyle{remark}
%\newtheorem*{rmk*}{Remark}
%\newtheorem*{rmks*}{Remarks}

%% This defines the "proo" environment, which is the same as proof, but
%% with "Proof:" instead of "Proof.". I prefer the former.
%\newenvironment{proo}{\begin{proof}[Proof:]}{\end{proof}}

%\usepackage[subfigure]{tocloft}
\usepackage[explicit]{titlesec}%
\usepackage{titletoc}
\usepackage{etoolbox}

% To add space between the Table of Contents, List of Figures and the List of Tables and the list content
\addtocontents{toc}{\vspace{1.2cm}}
\addtocontents{lof}{\vspace{1.2cm}}
\addtocontents{lot}{\vspace{1.2cm}}

% This command for chapters
\newcommand\chap[1]{%
  \chapter*{#1}%
  \addcontentsline{toc}{chapter}{#1}}

% Table of contents formatting
% Removing the dots between the Title and the page number
\makeatletter
\renewcommand{\@dotsep}{10000} 
\makeatother

\usepackage{tabto}
\usepackage{makebox}

% Table of contents font and space modifications

\titlecontents{chapter}[0pt]
    {\vskip 10pt \bfseries}
    {\bfseries\text{Chapter }\thecontentslabel\tabto{3.5cm}}
    {}
    {\hfill\bfseries\contentspage}

\titlecontents{section}[0pt]
    {}
    {\quad\quad\thecontentslabel\tabto{3.5cm}}
    {}
    {\hfill\contentspage}

\titlecontents{subsection}[0pt]
    {}
    {\quad\quad\thecontentslabel\tabto{3.5cm}}
    {}
    {\hfill\contentspage}
    
\titlecontents{table}[0pt]
    {}
    {\quad\quad\thecontentslabel\tabto{3.5cm}}
    {}
    {\hfill\contentspage}

\titlecontents{figure}[0pt]
    {}
    {\quad\quad\thecontentslabel\tabto{3.5cm}}
    {}
    {\hfill\contentspage}

% Change the Table of Contents, List of Tables ... etc. font size insited of big font
\renewcommand{\contentsname}{\normalsize{Table of Contents}}
\renewcommand{\listfigurename}{\normalsize{List of Figures}}
\renewcommand{\listtablename}{\normalsize{List of Tables}}
\renewcommand{\bibname}{\normalsize{Bibliography}}
\renewcommand{\indexname}{\normalsize{Index}}

% May 2009 added this to move page number up a bit
\addtolength{\voffset}{-2em}

% This is to format the chapter tags in this file
\usepackage{chngcntr}
\usepackage{lipsum}% just to generate text for the example

%% Page layout (customized to letter paper and Stevens requirements):
% if not using pdflatex to produce output, you may need to change to pagewidth and pageheight variables.
%\pagewidth 8.5in
%\pageheight 11in 
\pdfpagewidth 8.5in
\pdfpageheight 11in 
%
\setlength{\textheight}{8.5in} 
\setlength{\oddsidemargin}{0.5in}  
\setlength{\evensidemargin}{0.5in} 
\setlength{\textwidth}{6.0in}
\setlength{\topmargin}{0.in}    
\setlength{\headheight}{0.5in}
\setlength{\headwidth}{6.0in}
% change from .25in to .5 in May 2009 
\setlength{\headsep}{0.65in}
\setlength{\parindent}{12mm}

% For each chapter and section titles in the rest of the document the font formatting

% Chapter styles
\usepackage{sectsty}

\chapternumberfont{\normalsize} 
\chaptertitlefont{\normalsize}

\makeatletter
% No extra space between chapter number and chapter header lines:
\patchcmd{\@makechapterhead} {\vskip 20}{\vskip 0} {}{}
% Reduce extra space between chapter header and section header lines by 50%:
\patchcmd{\@makechapterhead} {\vskip 40}{\vskip 20}{}{}
\patchcmd{\@makeschapterhead}{\vskip 40}{\vskip 20}{}{} % for unnumbered chapters
\makeatother

% Sections styles
\sectionfont{\normalsize}

% Sub-sections styles
\subsectionfont{\normalsize}



%% Use the following commands, if desired, during production.
%% Comment them out for final version.
%\usepackage{layout} % defines the \layout command, see below
%\setlength{\hoffset}{-.75in} % creates a large right margin for notes and \showlabels

\pagestyle{fancy}
\fancyhf{}
% this prints a line under the header
\renewcommand{\headrulewidth}{0 pt}
%this prints a line under the footer
\renewcommand{\footrulewidth}{0 pt}
\fancyhead[RO]{}
\fancyhead[LO]{}
\fancyfoot[C]{}
\rhead{\thepage}

\fancypagestyle{plain}{%
\fancyhf{}
\rhead{\thepage}
}

%% Page layout (customized to letter paper and NYU requirements):
\setlength{\headheight}{20pt} 

%% Use the line below for official NYU version, which requires
%% double line spacing. For all other uses, this is unnecessary,
%% so the line can be commented out.
\onehalfspacing % requires package setspace, invoked above

%% Each of the following lines defines the \com command, which produces
%% a comment (notes for yourself, for instance) in the output file.
%% Example:    \com{this will appear as a comment in the output}
%% Choose (uncomment) only one of the three forms:
%\newcommand{\com}[1]{[/// {#1} ///]}       % between [/// and ///].
\newcommand{\com}[1]{\marginpar{\tiny #1}} % as (tiny) margin notes
%\newcommand{\com}[1]{}                     % suppress all comments.

%% Cross-referencing utilities. Use one or the other--whichever you prefer--
%% but comment out both lines for final version.
%\usepackage{showlabels}
%\usepackage{showkeys}
% \pagestyle{headings}

\begin{document}
%% Produces a test "layout" page, for "debugging" purposes only.
%% Comment out for final version.
%\layout % requires package layout (see above, on this same file)
%% Sets page numbering to "roman style" i, ii, iii, iv, etc:

%%%%%% Cover page %%%%%%%%%%%
%% Sets page numbering to "roman style" i, ii, iii, iv, etc:
\pagenumbering{roman}
\thispagestyle{empty}
\begin{center}
{
  {\thesistitle}
  \vspace{.15in}
  
    by
    
  \vspace{.15in}
  \thesisauthor
  
  \vspace{.15in}
  
 {A DISSERTATION}\\
  \vspace{.2in}
  \begin{spacing}{1}
    {Submitted to the Faculty of the Stevens Institute of Technology\\
    in partial fulfillment of the requirements for the degree of}
    \end{spacing}
  \vspace{.2in}
  
  {DOCTOR OF PHILOSOPHY}\\
  \vspace{1.0in}
  % for master thesis, change Chairman to Advisor
    \hfill 
    \begin{minipage}{80mm}
    \begin{spacing}{ }\noindent \rule{3.2in}{0.1mm}
        \thesisname, Candidate\\[3mm]
        \underline{ADVISORY COMMITTEE}\\[3mm]
        \noindent \rule{3.2in}{0.1mm}\\[-1.3mm]
        % for master thesis, change Chairman to Advisor
        {\thesischairadvisor}, Chairman  \hfill{Date}\\[2mm]
        {\noindent \rule{3.2in}{0.1mm}}\\[-1.3mm]
        {\committeenameA}        \hfill{Date}\\[2mm]
        {\noindent \rule{3.2in}{0.1mm}}\\[-1.3mm]
        {\committeenameB}        \hfill{Date}\\[2mm]
        {\noindent \rule{3.2in}{0.1mm}}\\[-1.3mm]
        {\committeenameC}        \hfill{Date}\\[2mm]
    \end{spacing}
  \end{minipage}
  \vfill
  
  {STEVENS INSTITUTE OF TECHNOLOGY\\
  \vspace{-0.05in}
  Castle Point on Hudson\\
  Hoboken, NJ 07030
  }
  % \vfill

  {\graddate}
}

\end{center}

\newpage



%%%%%%%%%%%%%% Microfilm / Publishing Page ProQuest %%%%%%%%%%%%%%%%%
% You can comment this section it is here to show you how it will appear when it submitted.
\thispagestyle{empty}
\begin{center}
ProQuest Number: XXXXXXXX

\vspace{.45in}

All rights reserved.

\vspace{.1in}

INFORMATION TO ALL USERS\\
The quality of this reproduction is dependent upon the quality of the copy submitted.
\vspace{.2in}

In the unlikely event that  the author did not send a complete manuscript and there are missing pages, these will be  noted. Also, if material had to be removed, a note will indicate the deletion.

\vspace{.1in}

\begin{figure}[H]
  \centering
  \includegraphics[width=200px]{misc/proquest-seeklogo.eps}
\end{figure}

\vspace{.1in}

ProQuest Number: XXXXXXXX

\vspace{.1in}

Published  by  ProQuest  LLC (\the\year).        Copyright of the Dissertation is held by the Author.

\vspace{.2in}

All rights reserved. This work is protected against  unauthorized copying under Title 17, United States Code Microform Edition {\textcopyright} ProQuest  LLC.

\vspace{.2in}

ProQuest LLC.\\
789 East Eisenhower Parkway\\
P.O. Box 1346\\
Ann Arbor, MI  48106-1346

\end{center}
\newpage

%%%%%% Copyrights page %%%%%%%%%%%
%
\setcounter{page}{2}
%% No numbering in the title page:
\thispagestyle{empty}
%
\begin{center}
{
  \vspace*{\fill}

  {\textcopyright} \graddate, \thesisname. All rights reserved.
}

\end{center}

\newpage
\doublespacing

%%%%%%%%%%%%%% Abstract %%%%%%%%%%%%%%%%%
\begin{center}
    {\thesistitle}\\ 
    {ABSTRACT}\\
    \vspace{.05in}
\end{center}
\addcontentsline{toc}{chapter}{Abstract}
%!TEX root = thesis.tex

%
\lipsum[2]
\vspace{0.4in}
\begin{flushleft}
Author: \thesisname \\
Advisor: \thesischairadvisor \\
Date: \thesisdate \\
Department: \thesisdepartment \\
Degree: \thesisdegree \\
\end{flushleft}
\newpage

%%%%%%%%%%%%%% Dedication Page %%%%%%%%%%%%%%%%%
%% Comment out the following lines if you do not want to dedicate it is optional
\chapter*{Dedication}
\addcontentsline{toc}{chapter}{Dedication}
\strut \vspace{2in}
\begin{center}
This dissertation is dedicated to my family, who supported me during my Ph.D. journey.
    \end{center}
    \vfill \strut
    \newpage
\newpage

%%%%%%%%%%%%%% Acknowledgements %%%%%%%%%%%%%%%%%
%% Comment out the following lines if you do not want to acknowledge
%% anyone's help...
\addcontentsline{toc}{chapter}{Acknowledgments}
%!TEX root = thesis.tex

%% Write your acknowledgements in this file. If you do not want to acknowledge anyone,
%% you can delete this file and comment out the corresponding part in the "thesis.tex"
%% file.
\noindent \textbf{Acknowledgments} \\[6mm]
The realization of this work was only possible due to the several people’s collaboration, which desires me to express my gratefulness.

\lipsum[2]

    \vfill \strut


\newpage

%%%% Table of Contents %%%%%%%%%%%%
\setcounter{tocdepth}{2} % To show a three level depth of sections

\tableofcontents

% \clearpage
% \pagestyle{headings}
\newpage

%%%%% List of Tables %%%%%%%%%%%%%
%% Comment out the following two lines if your thesis does not
%% contain any tables. The list of tables contains only
%% those tables included withing the "table" environment.
\listoftables
\newpage

%%%%% List of Figures %%%%%%%%%%%%%
%% Comment out the following two lines if your thesis does not
%% contain any figures. The list of figures contains only
%% those figures included withing the "figure" environment.
\listoffigures\addcontentsline{toc}{chapter}{\listfigurename}
\newpage

%%%%% Body of thesis starts %%%%%%%%%%%%
\pagenumbering{arabic} % switches page numbering to arabic: 1, 2, 3, etc.

%% Introduction. If your thesis has no introduction, or chapter 1 is
%% meant to be the introduction, then comment out the lines below.
%% \section*{Introduction}\addcontentsline{toc}{section}{Introduction}
%\input{intro}

%%If your thesis has different "Parts", use commands such as the following:

\chapter{Introduction}

\lipsum[2]

% At the end of each chapter
\newpage

\chapter{Two} \label{CH2}

\lipsum[2]

% At the end of each chapter
\newpage

\chapter{Three} \label{CH3}

\lipsum[2]

% At the end of each chapter
\newpage

\chapter{Four} \label{CH4}

\lipsum[2]

% At the end of each chapter
\newpage

\chapter{Five} \label{CH5}

\lipsum[2]

% At the end of each chapter
\newpage

\chapter{Six} \label{CH6}

\lipsum[2]

% At the end of each chapter
\newpage

%!TEX root = ../thesis.tex
\chapter{Conclusion and Future Works} \label{conc}

\lipsum[2]

% At the end of each chapter
\newpage
%%%%% Appendices start %%%%%%%%%%%%%%%%
%% Comment out the following line if your thesis has no appendix
\addcontentsline{toc}{chapter}{Appendices}
\noindent \textbf{Appendices}
\appendix
\addcontentsline{toc}{section}{Appendix A}  %% removed \\
\noindent \textbf{Appendix A}
\vspace{12pt}

\noindent Appendices at the end of a dissertation are optional, and depend on the content of the dissertation. There can be one or more appendicies, however they should retain the page numbering requirements for dissertations.  Any concerns about the formatting of an appendix should be brought to Doris Oliver, who can direct you how to format your appendix if you have questions.

\begin{center}
\begin{tabular}{|l|c|p{3.0in}|}
\hline
\multicolumn{3}{|c|}{Theoretical Dissertation Timeline}\\ \hline
Taskt & Time to Finish & Notes\\ \hline
Problem statement & 10 hours & Initially very upbeat.\\ \hline
Research & 3 days&Literature search to very previous studies.\\ \hline
Reformulation&4 hours&Presented and accepted by advisor\\ \hline
Research&20 days&Literature search to very previous  studies.\\ \hline
Experiments&14 days&Do some experiments and get results.\\ \hline
Format&1 day&Understand format guidelines for paper.\\ \hline
Write&years&Write the paper.\\ \hline
Revise&not too long&Proof read, etc.\\ \hline
Format&1-3 days&Verify correct report format is used.\\ \hline
See Library&1 hour&Meet with Doris to verify formatting.\\ \hline
Defend&1 day&Defend your research.\\ \hline
Revise&0 hours&It was perfect the first time.\\ \hline
Submit&1 day&Submit final dissertation to the library.\\ \hline
\end{tabular}
\end{center}
\newpage

\addcontentsline{toc}{section}{Appendix B}  %% removed \\
\noindent \textbf{Appendix B}
\vspace{12pt}

\noindent Another one! Here is more text to go.
\newpage



%% Note: If your thesis has more than one appendix, NYU requires a "list of
%% appendices" page before the body of the thesis. I don't provide the tools
%% to create that here, so you're on your own for that one... Sorry.
%\addcontentsline{toc}{section}{Appendix B}  %% removed \\
\noindent \textbf{Appendix B}
\vspace{12pt}

\noindent Another one! Here is more text to go.
\newpage



%%%% Input bibliography file %%%%%%%%%%%%%%%
% % NYU PhD thesis format. Original template created by José Koiller 2007--2008.
%% Updated by Anshul Vikram Pandey with new design guidelines. 2017--2018.
%%% Modified by Abdullah Khanfor for Stevens Institute of Technology PhD thesis format design guidelines 2019--2020. Link to the old template: https://github.com/khanfor/stevens-phd-thesis-dissertation-template
%%% Updated by Mofadal Alymani for Stevens Institute of Technology PhD thesis format design guidelines 2020--2021.

%% Use the first of the following lines during production to
%% easily spot "overfull boxes" in the output. Use the second
%% line for the final version.
% \documentclass[12pt,draft,letterpaper]{report}
% \documentclass[12pt,letterpaper]{report}
\documentclass[12pt]{report}
\usepackage{siunitx}
\usepackage{enumitem}
\usepackage{url}
\usepackage[breaklinks]{hyperref}
\def\UrlBreaks{\do\/\do-}
%% Replace the title, name, advisor name, graduation date and dedication below with
%% your own. Graduation months must be January, May or September.
\newcommand{\thesistitle}{Thesis Title}
\newcommand{\thesisauthor}{First Last}
\newcommand{\thesisadvisor}{Advisor}
\newcommand{\thesisyear}{2021}
\newcommand{\thesisname}{First Last}
\newcommand{\thesischairadvisor}{Dr. Advisor}    % this name prints on the title page as chairman and the abstract page as advisor
\newcommand{\committeenameA}{Dr. Member}
\newcommand{\committeenameB}{Dr. Member}
\newcommand{\committeenameC}{Dr. Member}
\newcommand{\thesisdepartment}{Department}
\newcommand{\thesisdate}{April 26, 2021}
\newcommand{\thesistype}{dissertation}
\newcommand{\thesisdegree}{Doctor of Philosophy}
\newcommand{\graddate}{\the\year} % like 2020, 2019, no month or day should be written
\newcommand{\thesissigline}[1]{%
  \leftline{\hbox to 2.5in{}\hrulefill}
  \endgraf
  \vspace*{-18pt}
  \leftline{\hbox to 2.53in{}{#1}}}

%% If you do not want a dedication, scroll down and comment out
%% the appropriate lines in this file.
\newcommand{\thesisdedication}{To all the Ph.D. pursuing brave souls}

%% The following makes chapters and sections, but not subsections,
%% appear in the TOC (table of contents). Increase to 2 or 3 to
%% make subsections or subsubsections appear, respectively. It seems
%% to be usual to use the "1" setting, however.
\setcounter{tocdepth}{1}

%% Sectional units up to subsubsections are numbered. To number
%% subsections, but not subsubsections, decrease this counter to 2.
\setcounter{secnumdepth}{3}

% Setting a gap between page number and text block

%% This inputs your auxiliary file with \usepackage's and \newcommand's:
%% It is assumed that that file is called "definitions.tex".
%%
%% Place here your \usepackage's. Some recommended packages are already included.
%%

% Graphics:
\usepackage[final]{graphicx}
%\usepackage{graphicx} % use this line instead of the above to suppress graphics in draft copies
%\usepackage{graphpap} % \defines the \graphpaper command

% Indent first line of each section:
%\usepackage{indentfirst}

% Good AMS stuff:
\usepackage{amsthm} % facilities for theorem-like environments
\usepackage[tbtags]{amsmath} % a lot of good stuff!

% Fonts and symbols:
\usepackage{amsfonts}
\usepackage{amssymb}

\usepackage{xspace}

% Algorithm figures
\usepackage{algorithm}
\usepackage[noend]{algpseudocode}

\usepackage{microtype}
\usepackage{subfigure}
\usepackage{color}
\usepackage{todonotes}
\usepackage{url}
\newfloat{algorithm}{t}{lop}

\usepackage{blindtext}

%% Controls spacing between lines (\doublespacing, \onehalfspacing, etc.):
\usepackage[utf8x]{inputenc}
\usepackage{fancyhdr}

% This package to change the font if the document. This font is optional as you preference. You can comment it to use the CMU font
%\usepackage{helvet}
%\renewcommand{\familydefault}{\sfdefault}

%% \usepackage{amsmath}
%% \usepackage{amssymb}
\usepackage{lipsum}
% \newfloat{algorithm}{t}{lop}

% Packages for setting the length and width of document
\usepackage{setspace}

% Package for sideway images and figures
\usepackage{rotating}
\usepackage{pdflscape}

\usepackage{wrapfig}
\usepackage{lscape}
\usepackage{epstopdf}

\usepackage{caption}

% Formatting tools:
%\usepackage{relsize} % relative font size selection, provides commands \textsmalle, \textlarger
%\usepackage{xspace} % gentle spacing in macros, such as \newcommand{\acims}{\textsc{acim}s\xspace}

% Page formatting utility:
%\usepackage{geometry}
\usepackage{multirow}

% For MATLAB code
\usepackage{listings}
\usepackage[framed,numbered,autolinebreaks,useliterate]{mcode}

% For citations
\usepackage[numbers,sort]{natbib}
\usepackage[nottoc]{tocbibind}

\usepackage[all,cmtip]{xy}

% Change the color of the hyperlinks and titles from blue to black
\usepackage{hyperref}
\hypersetup{
    colorlinks = false,
    linkbordercolor = {white},
    linkcolor=black,
    filecolor=black,
    urlcolor=black,
    citecolor=black
}

%%
%% Place here your \newcommand's and \renewcommand's. Some examples already included.
%%
%\newcommand{\acims}{\textsc{acim}s\xspace}
\newcommand{\Mspace}        {{\mathbb M}}
\newcommand{\Rspace}        {{\mathbb R}}
\newcommand{\Cspace}        {{\mathbb C}}

\newcommand{\Mo}        {{\hat M}}
\newcommand{\Ms}        {{\tilde M}}
\newcommand{\Do}          {{\hat D}}
\newcommand{\Ds}        {{\tilde D}}
\newcommand{\doo}          {{\hat d}}
\newcommand{\dss}        {{\tilde d}}
\newcommand{\w}        {{\mathbf w}}

% general
\newcommand{\ie}{i.e.}
\newcommand{\eg}{e.g.}
\newcommand{\reffig}[1]{{Figure~\ref{#1}}}
\newcommand{\refchap}[1]{{Chapter~\ref{#1}}}
\newcommand{\refsec}[1]{{Section~\ref{#1}}}
\newcommand{\reftab}[1]{{Table~\ref{#1}}}
\newcommand{\refapp}[1]{{Appendix~\ref{#1}}}
\newcommand{\refeq}[1]{{Equation~\ref{#1}}}
\newcommand{\refalg}[1]{{Algorithm~\ref{#1}}}
\newcommand{\myparagraph}[1]{\noindent \textbf{#1}}
\newcommand{\highlight}[1]{{\color{black}#1}}

%%
%% Place here your \newtheorem's:
%%

%% Some examples commented out below. Create your own or use these...
%%%%%%%%%\swapnumbers % this makes the numbers appear before the statement name.
%\theoremstyle{plain}
%\newtheorem{thm}{Theorem}[chapter]
%\newtheorem{prop}[thm]{Proposition}
%\newtheorem{lemma}[thm]{Lemma}
%\newtheorem{cor}[thm]{Corollary}

%\theoremstyle{definition}
%\newtheorem{define}{Definition}[chapter]

%\theoremstyle{remark}
%\newtheorem*{rmk*}{Remark}
%\newtheorem*{rmks*}{Remarks}

%% This defines the "proo" environment, which is the same as proof, but
%% with "Proof:" instead of "Proof.". I prefer the former.
%\newenvironment{proo}{\begin{proof}[Proof:]}{\end{proof}}

%\usepackage[subfigure]{tocloft}
\usepackage[explicit]{titlesec}%
\usepackage{titletoc}
\usepackage{etoolbox}

% To add space between the Table of Contents, List of Figures and the List of Tables and the list content
\addtocontents{toc}{\vspace{1.2cm}}
\addtocontents{lof}{\vspace{1.2cm}}
\addtocontents{lot}{\vspace{1.2cm}}

% This command for chapters
\newcommand\chap[1]{%
  \chapter*{#1}%
  \addcontentsline{toc}{chapter}{#1}}

% Table of contents formatting
% Removing the dots between the Title and the page number
\makeatletter
\renewcommand{\@dotsep}{10000} 
\makeatother

\usepackage{tabto}
\usepackage{makebox}

% Table of contents font and space modifications

\titlecontents{chapter}[0pt]
    {\vskip 10pt \bfseries}
    {\bfseries\text{Chapter }\thecontentslabel\tabto{3.5cm}}
    {}
    {\hfill\bfseries\contentspage}

\titlecontents{section}[0pt]
    {}
    {\quad\quad\thecontentslabel\tabto{3.5cm}}
    {}
    {\hfill\contentspage}

\titlecontents{subsection}[0pt]
    {}
    {\quad\quad\thecontentslabel\tabto{3.5cm}}
    {}
    {\hfill\contentspage}
    
\titlecontents{table}[0pt]
    {}
    {\quad\quad\thecontentslabel\tabto{3.5cm}}
    {}
    {\hfill\contentspage}

\titlecontents{figure}[0pt]
    {}
    {\quad\quad\thecontentslabel\tabto{3.5cm}}
    {}
    {\hfill\contentspage}

% Change the Table of Contents, List of Tables ... etc. font size insited of big font
\renewcommand{\contentsname}{\normalsize{Table of Contents}}
\renewcommand{\listfigurename}{\normalsize{List of Figures}}
\renewcommand{\listtablename}{\normalsize{List of Tables}}
\renewcommand{\bibname}{\normalsize{Bibliography}}
\renewcommand{\indexname}{\normalsize{Index}}

% May 2009 added this to move page number up a bit
\addtolength{\voffset}{-2em}

% This is to format the chapter tags in this file
\usepackage{chngcntr}
\usepackage{lipsum}% just to generate text for the example

%% Page layout (customized to letter paper and Stevens requirements):
% if not using pdflatex to produce output, you may need to change to pagewidth and pageheight variables.
%\pagewidth 8.5in
%\pageheight 11in 
\pdfpagewidth 8.5in
\pdfpageheight 11in 
%
\setlength{\textheight}{8.5in} 
\setlength{\oddsidemargin}{0.5in}  
\setlength{\evensidemargin}{0.5in} 
\setlength{\textwidth}{6.0in}
\setlength{\topmargin}{0.in}    
\setlength{\headheight}{0.5in}
\setlength{\headwidth}{6.0in}
% change from .25in to .5 in May 2009 
\setlength{\headsep}{0.65in}
\setlength{\parindent}{12mm}

% For each chapter and section titles in the rest of the document the font formatting

% Chapter styles
\usepackage{sectsty}

\chapternumberfont{\normalsize} 
\chaptertitlefont{\normalsize}

\makeatletter
% No extra space between chapter number and chapter header lines:
\patchcmd{\@makechapterhead} {\vskip 20}{\vskip 0} {}{}
% Reduce extra space between chapter header and section header lines by 50%:
\patchcmd{\@makechapterhead} {\vskip 40}{\vskip 20}{}{}
\patchcmd{\@makeschapterhead}{\vskip 40}{\vskip 20}{}{} % for unnumbered chapters
\makeatother

% Sections styles
\sectionfont{\normalsize}

% Sub-sections styles
\subsectionfont{\normalsize}



%% Use the following commands, if desired, during production.
%% Comment them out for final version.
%\usepackage{layout} % defines the \layout command, see below
%\setlength{\hoffset}{-.75in} % creates a large right margin for notes and \showlabels

\pagestyle{fancy}
\fancyhf{}
% this prints a line under the header
\renewcommand{\headrulewidth}{0 pt}
%this prints a line under the footer
\renewcommand{\footrulewidth}{0 pt}
\fancyhead[RO]{}
\fancyhead[LO]{}
\fancyfoot[C]{}
\rhead{\thepage}

\fancypagestyle{plain}{%
\fancyhf{}
\rhead{\thepage}
}

%% Page layout (customized to letter paper and NYU requirements):
\setlength{\headheight}{20pt} 

%% Use the line below for official NYU version, which requires
%% double line spacing. For all other uses, this is unnecessary,
%% so the line can be commented out.
\onehalfspacing % requires package setspace, invoked above

%% Each of the following lines defines the \com command, which produces
%% a comment (notes for yourself, for instance) in the output file.
%% Example:    \com{this will appear as a comment in the output}
%% Choose (uncomment) only one of the three forms:
%\newcommand{\com}[1]{[/// {#1} ///]}       % between [/// and ///].
\newcommand{\com}[1]{\marginpar{\tiny #1}} % as (tiny) margin notes
%\newcommand{\com}[1]{}                     % suppress all comments.

%% Cross-referencing utilities. Use one or the other--whichever you prefer--
%% but comment out both lines for final version.
%\usepackage{showlabels}
%\usepackage{showkeys}
% \pagestyle{headings}

\begin{document}
%% Produces a test "layout" page, for "debugging" purposes only.
%% Comment out for final version.
%\layout % requires package layout (see above, on this same file)
%% Sets page numbering to "roman style" i, ii, iii, iv, etc:

%%%%%% Cover page %%%%%%%%%%%
%% Sets page numbering to "roman style" i, ii, iii, iv, etc:
\pagenumbering{roman}
\thispagestyle{empty}
\begin{center}
{
  {\thesistitle}
  \vspace{.15in}
  
    by
    
  \vspace{.15in}
  \thesisauthor
  
  \vspace{.15in}
  
 {A DISSERTATION}\\
  \vspace{.2in}
  \begin{spacing}{1}
    {Submitted to the Faculty of the Stevens Institute of Technology\\
    in partial fulfillment of the requirements for the degree of}
    \end{spacing}
  \vspace{.2in}
  
  {DOCTOR OF PHILOSOPHY}\\
  \vspace{1.0in}
  % for master thesis, change Chairman to Advisor
    \hfill 
    \begin{minipage}{80mm}
    \begin{spacing}{ }\noindent \rule{3.2in}{0.1mm}
        \thesisname, Candidate\\[3mm]
        \underline{ADVISORY COMMITTEE}\\[3mm]
        \noindent \rule{3.2in}{0.1mm}\\[-1.3mm]
        % for master thesis, change Chairman to Advisor
        {\thesischairadvisor}, Chairman  \hfill{Date}\\[2mm]
        {\noindent \rule{3.2in}{0.1mm}}\\[-1.3mm]
        {\committeenameA}        \hfill{Date}\\[2mm]
        {\noindent \rule{3.2in}{0.1mm}}\\[-1.3mm]
        {\committeenameB}        \hfill{Date}\\[2mm]
        {\noindent \rule{3.2in}{0.1mm}}\\[-1.3mm]
        {\committeenameC}        \hfill{Date}\\[2mm]
    \end{spacing}
  \end{minipage}
  \vfill
  
  {STEVENS INSTITUTE OF TECHNOLOGY\\
  \vspace{-0.05in}
  Castle Point on Hudson\\
  Hoboken, NJ 07030
  }
  % \vfill

  {\graddate}
}

\end{center}

\newpage



%%%%%%%%%%%%%% Microfilm / Publishing Page ProQuest %%%%%%%%%%%%%%%%%
% You can comment this section it is here to show you how it will appear when it submitted.
\thispagestyle{empty}
\begin{center}
ProQuest Number: XXXXXXXX

\vspace{.45in}

All rights reserved.

\vspace{.1in}

INFORMATION TO ALL USERS\\
The quality of this reproduction is dependent upon the quality of the copy submitted.
\vspace{.2in}

In the unlikely event hat  the author did not send a complete manuscript and there are missing pages, these will be  noted. Also ,if material had to be removed, a note will indicate the deletion.

\vspace{.1in}

\begin{figure}[H]
  \centering
  \includegraphics[width=200px]{misc/proquest-seeklogo.eps}
\end{figure}

\vspace{.1in}

ProQuest Number: XXXXXXXX

\vspace{.1in}

Published  by  ProQuest  LLC (\the\year).        Copyright of the Dissertation is held by the Author.

\vspace{.2in}

All rights reserved. This work is protected against  unauthorized copying under Title 17, United States Code Microform Edition {\textcopyright} ProQuest  LLC.

\vspace{.2in}

ProQuest LLC.\\
789 East Eisenhower Parkway\\
P.O. Box 1346\\
Ann Arbor, MI  48106-1346

\end{center}
\newpage

%%%%%% Copyrights page %%%%%%%%%%%
%
\setcounter{page}{2}
%% No numbering in the title page:
\thispagestyle{empty}
%
\begin{center}
{
  \vspace*{\fill}

  {\textcopyright} \graddate, \thesisname. All rights reserved.
}

\end{center}

\newpage
\doublespacing

%%%%%%%%%%%%%% Abstract %%%%%%%%%%%%%%%%%
\begin{center}
    {\thesistitle}\\ 
    {ABSTRACT}\\
    \vspace{.05in}
\end{center}
\addcontentsline{toc}{chapter}{Abstract}
%!TEX root = thesis.tex

%
\lipsum[2]
\vspace{0.4in}
\begin{flushleft}
Author: \thesisname \\
Advisor: \thesischairadvisor \\
Date: \thesisdate \\
Department: \thesisdepartment \\
Degree: \thesisdegree \\
\end{flushleft}
\newpage

%%%%%%%%%%%%%% Dedication Page %%%%%%%%%%%%%%%%%
%% Comment out the following lines if you do not want to dedicate it is optional
\chapter*{Dedication}
\addcontentsline{toc}{chapter}{Dedication}
\strut \vspace{2in}
\begin{center}
This dissertation is dedicated to my family, who supported me during my Ph.D. journey.
    \end{center}
    \vfill \strut
    \newpage
\newpage

%%%%%%%%%%%%%% Acknowledgements %%%%%%%%%%%%%%%%%
%% Comment out the following lines if you do not want to acknowledge
%% anyone's help...
\addcontentsline{toc}{chapter}{Acknowledgements}
%!TEX root = thesis.tex

%% Write your acknowledgements in this file. If you do not want to acknowledge anyone,
%% you can delete this file and comment out the corresponding part in the "thesis.tex"
%% file.
\noindent \textbf{Acknowledgments} \\[6mm]
The realization of this work was only possible due to the several people’s collaboration, which desires me to express my gratefulness.

\lipsum[2]

    \vfill \strut


\newpage

%%%% Table of Contents %%%%%%%%%%%%
\setcounter{tocdepth}{2} % To show a three level depth of sections

\tableofcontents

% \clearpage
% \pagestyle{headings}
\newpage

%%%%% List of Tables %%%%%%%%%%%%%
%% Comment out the following two lines if your thesis does not
%% contain any tables. The list of tables contains only
%% those tables included withing the "table" environment.
\listoftables
\newpage

%%%%% List of Figures %%%%%%%%%%%%%
%% Comment out the following two lines if your thesis does not
%% contain any figures. The list of figures contains only
%% those figures included withing the "figure" environment.
\listoffigures\addcontentsline{toc}{chapter}{\listfigurename}
\newpage

%%%%% Body of thesis starts %%%%%%%%%%%%
\pagenumbering{arabic} % switches page numbering to arabic: 1, 2, 3, etc.

%% Introduction. If your thesis has no introduction, or chapter 1 is
%% meant to be the introduction, then comment out the lines below.
%% \section*{Introduction}\addcontentsline{toc}{section}{Introduction}
%\input{intro}

%%If your thesis has different "Parts", use commands such as the following:

\chapter{Introduction}

\lipsum[2]

% At the end of each chapter
\newpage

\chapter{Two} \label{CH2}

\lipsum[2]

% At the end of each chapter
\newpage

\chapter{Three} \label{CH3}

\lipsum[2]

% At the end of each chapter
\newpage

\chapter{Four} \label{CH4}

\lipsum[2]

% At the end of each chapter
\newpage

\chapter{Five} \label{CH5}

\lipsum[2]

% At the end of each chapter
\newpage

\chapter{Six} \label{CH6}

\lipsum[2]

% At the end of each chapter
\newpage

%!TEX root = ../thesis.tex
\chapter{Conclusion and Future Works} \label{conc}

\lipsum[2]

% At the end of each chapter
\newpage
%%%%% Appendices start %%%%%%%%%%%%%%%%
%% Comment out the following line if your thesis has no appendix
\addcontentsline{toc}{chapter}{Appendices}
\noindent \textbf{Appendices}
\appendix
\addcontentsline{toc}{section}{Appendix A}  %% removed \\
\noindent \textbf{Appendix A}
\vspace{12pt}

\noindent Appendices at the end of a dissertation are optional, and depend on the content of the dissertation. There can be one or more appendicies, however they should retain the page numbering requirements for dissertations.  Any concerns about the formatting of an appendix should be brought to Doris Oliver, who can direct you how to format your appendix if you have questions.

\begin{center}
\begin{tabular}{|l|c|p{3.0in}|}
\hline
\multicolumn{3}{|c|}{Theoretical Dissertation Timeline}\\ \hline
Taskt & Time to Finish & Notes\\ \hline
Problem statement & 10 hours & Initially very upbeat.\\ \hline
Research & 3 days&Literature search to very previous studies.\\ \hline
Reformulation&4 hours&Presented and accepted by advisor\\ \hline
Research&20 days&Literature search to very previous  studies.\\ \hline
Experiments&14 days&Do some experiments and get results.\\ \hline
Format&1 day&Understand format guidelines for paper.\\ \hline
Write&years&Write the paper.\\ \hline
Revise&not too long&Proof read, etc.\\ \hline
Format&1-3 days&Verify correct report format is used.\\ \hline
See Library&1 hour&Meet with Doris to verify formatting.\\ \hline
Defend&1 day&Defend your research.\\ \hline
Revise&0 hours&It was perfect the first time.\\ \hline
Submit&1 day&Submit final dissertation to the library.\\ \hline
\end{tabular}
\end{center}
\newpage

\addcontentsline{toc}{section}{Appendix B}  %% removed \\
\noindent \textbf{Appendix B}
\vspace{12pt}

\noindent Another one! Here is more text to go.
\newpage



%% Note: If your thesis has more than one appendix, NYU requires a "list of
%% appendices" page before the body of the thesis. I don't provide the tools
%% to create that here, so you're on your own for that one... Sorry.
%\addcontentsline{toc}{section}{Appendix B}  %% removed \\
\noindent \textbf{Appendix B}
\vspace{12pt}

\noindent Another one! Here is more text to go.
\newpage



%%%% Input bibliography file %%%%%%%%%%%%%%%
% % NYU PhD thesis format. Original template created by José Koiller 2007--2008.
%% Updated by Anshul Vikram Pandey with new design guidelines. 2017--2018.
%%% Modified by Abdullah Khanfor for Stevens Institute of Technology PhD thesis format design guidelines 2019--2020. Link to the old template: https://github.com/khanfor/stevens-phd-thesis-dissertation-template
%%% Updated by Mofadal Alymani for Stevens Institute of Technology PhD thesis format design guidelines 2020--2021.

%% Use the first of the following lines during production to
%% easily spot "overfull boxes" in the output. Use the second
%% line for the final version.
% \documentclass[12pt,draft,letterpaper]{report}
% \documentclass[12pt,letterpaper]{report}
\documentclass[12pt]{report}
\usepackage{siunitx}
\usepackage{enumitem}
\usepackage{url}
\usepackage[breaklinks]{hyperref}
\def\UrlBreaks{\do\/\do-}
%% Replace the title, name, advisor name, graduation date and dedication below with
%% your own. Graduation months must be January, May or September.
\newcommand{\thesistitle}{Thesis Title}
\newcommand{\thesisauthor}{First Last}
\newcommand{\thesisadvisor}{Advisor}
\newcommand{\thesisyear}{2021}
\newcommand{\thesisname}{First Last}
\newcommand{\thesischairadvisor}{Dr. Advisor}    % this name prints on the title page as chairman and the abstract page as advisor
\newcommand{\committeenameA}{Dr. Member}
\newcommand{\committeenameB}{Dr. Member}
\newcommand{\committeenameC}{Dr. Member}
\newcommand{\thesisdepartment}{Department}
\newcommand{\thesisdate}{April 26, 2021}
\newcommand{\thesistype}{dissertation}
\newcommand{\thesisdegree}{Doctor of Philosophy}
\newcommand{\graddate}{\the\year} % like 2020, 2019, no month or day should be written
\newcommand{\thesissigline}[1]{%
  \leftline{\hbox to 2.5in{}\hrulefill}
  \endgraf
  \vspace*{-18pt}
  \leftline{\hbox to 2.53in{}{#1}}}

%% If you do not want a dedication, scroll down and comment out
%% the appropriate lines in this file.
\newcommand{\thesisdedication}{To all the Ph.D. pursuing brave souls}

%% The following makes chapters and sections, but not subsections,
%% appear in the TOC (table of contents). Increase to 2 or 3 to
%% make subsections or subsubsections appear, respectively. It seems
%% to be usual to use the "1" setting, however.
\setcounter{tocdepth}{1}

%% Sectional units up to subsubsections are numbered. To number
%% subsections, but not subsubsections, decrease this counter to 2.
\setcounter{secnumdepth}{3}

% Setting a gap between page number and text block

%% This inputs your auxiliary file with \usepackage's and \newcommand's:
%% It is assumed that that file is called "definitions.tex".
%%
%% Place here your \usepackage's. Some recommended packages are already included.
%%

% Graphics:
\usepackage[final]{graphicx}
%\usepackage{graphicx} % use this line instead of the above to suppress graphics in draft copies
%\usepackage{graphpap} % \defines the \graphpaper command

% Indent first line of each section:
%\usepackage{indentfirst}

% Good AMS stuff:
\usepackage{amsthm} % facilities for theorem-like environments
\usepackage[tbtags]{amsmath} % a lot of good stuff!

% Fonts and symbols:
\usepackage{amsfonts}
\usepackage{amssymb}

\usepackage{xspace}

% Algorithm figures
\usepackage{algorithm}
\usepackage[noend]{algpseudocode}

\usepackage{microtype}
\usepackage{subfigure}
\usepackage{color}
\usepackage{todonotes}
\usepackage{url}
\newfloat{algorithm}{t}{lop}

\usepackage{blindtext}

%% Controls spacing between lines (\doublespacing, \onehalfspacing, etc.):
\usepackage[utf8x]{inputenc}
\usepackage{fancyhdr}

% This package to change the font if the document. This font is optional as you preference. You can comment it to use the CMU font
%\usepackage{helvet}
%\renewcommand{\familydefault}{\sfdefault}

%% \usepackage{amsmath}
%% \usepackage{amssymb}
\usepackage{lipsum}
% \newfloat{algorithm}{t}{lop}

% Packages for setting the length and width of document
\usepackage{setspace}

% Package for sideway images and figures
\usepackage{rotating}
\usepackage{pdflscape}

\usepackage{wrapfig}
\usepackage{lscape}
\usepackage{epstopdf}

\usepackage{caption}

% Formatting tools:
%\usepackage{relsize} % relative font size selection, provides commands \textsmalle, \textlarger
%\usepackage{xspace} % gentle spacing in macros, such as \newcommand{\acims}{\textsc{acim}s\xspace}

% Page formatting utility:
%\usepackage{geometry}
\usepackage{multirow}

% For MATLAB code
\usepackage{listings}
\usepackage[framed,numbered,autolinebreaks,useliterate]{mcode}

% For citations
\usepackage[numbers,sort]{natbib}
\usepackage[nottoc]{tocbibind}

\usepackage[all,cmtip]{xy}

% Change the color of the hyperlinks and titles from blue to black
\usepackage{hyperref}
\hypersetup{
    colorlinks = false,
    linkbordercolor = {white},
    linkcolor=black,
    filecolor=black,
    urlcolor=black,
    citecolor=black
}

%%
%% Place here your \newcommand's and \renewcommand's. Some examples already included.
%%
%\newcommand{\acims}{\textsc{acim}s\xspace}
\newcommand{\Mspace}        {{\mathbb M}}
\newcommand{\Rspace}        {{\mathbb R}}
\newcommand{\Cspace}        {{\mathbb C}}

\newcommand{\Mo}        {{\hat M}}
\newcommand{\Ms}        {{\tilde M}}
\newcommand{\Do}          {{\hat D}}
\newcommand{\Ds}        {{\tilde D}}
\newcommand{\doo}          {{\hat d}}
\newcommand{\dss}        {{\tilde d}}
\newcommand{\w}        {{\mathbf w}}

% general
\newcommand{\ie}{i.e.}
\newcommand{\eg}{e.g.}
\newcommand{\reffig}[1]{{Figure~\ref{#1}}}
\newcommand{\refchap}[1]{{Chapter~\ref{#1}}}
\newcommand{\refsec}[1]{{Section~\ref{#1}}}
\newcommand{\reftab}[1]{{Table~\ref{#1}}}
\newcommand{\refapp}[1]{{Appendix~\ref{#1}}}
\newcommand{\refeq}[1]{{Equation~\ref{#1}}}
\newcommand{\refalg}[1]{{Algorithm~\ref{#1}}}
\newcommand{\myparagraph}[1]{\noindent \textbf{#1}}
\newcommand{\highlight}[1]{{\color{black}#1}}

%%
%% Place here your \newtheorem's:
%%

%% Some examples commented out below. Create your own or use these...
%%%%%%%%%\swapnumbers % this makes the numbers appear before the statement name.
%\theoremstyle{plain}
%\newtheorem{thm}{Theorem}[chapter]
%\newtheorem{prop}[thm]{Proposition}
%\newtheorem{lemma}[thm]{Lemma}
%\newtheorem{cor}[thm]{Corollary}

%\theoremstyle{definition}
%\newtheorem{define}{Definition}[chapter]

%\theoremstyle{remark}
%\newtheorem*{rmk*}{Remark}
%\newtheorem*{rmks*}{Remarks}

%% This defines the "proo" environment, which is the same as proof, but
%% with "Proof:" instead of "Proof.". I prefer the former.
%\newenvironment{proo}{\begin{proof}[Proof:]}{\end{proof}}

%\usepackage[subfigure]{tocloft}
\usepackage[explicit]{titlesec}%
\usepackage{titletoc}
\usepackage{etoolbox}

% To add space between the Table of Contents, List of Figures and the List of Tables and the list content
\addtocontents{toc}{\vspace{1.2cm}}
\addtocontents{lof}{\vspace{1.2cm}}
\addtocontents{lot}{\vspace{1.2cm}}

% This command for chapters
\newcommand\chap[1]{%
  \chapter*{#1}%
  \addcontentsline{toc}{chapter}{#1}}

% Table of contents formatting
% Removing the dots between the Title and the page number
\makeatletter
\renewcommand{\@dotsep}{10000} 
\makeatother

\usepackage{tabto}
\usepackage{makebox}

% Table of contents font and space modifications

\titlecontents{chapter}[0pt]
    {\vskip 10pt \bfseries}
    {\bfseries\text{Chapter }\thecontentslabel\tabto{3.5cm}}
    {}
    {\hfill\bfseries\contentspage}

\titlecontents{section}[0pt]
    {}
    {\quad\quad\thecontentslabel\tabto{3.5cm}}
    {}
    {\hfill\contentspage}

\titlecontents{subsection}[0pt]
    {}
    {\quad\quad\thecontentslabel\tabto{3.5cm}}
    {}
    {\hfill\contentspage}
    
\titlecontents{table}[0pt]
    {}
    {\quad\quad\thecontentslabel\tabto{3.5cm}}
    {}
    {\hfill\contentspage}

\titlecontents{figure}[0pt]
    {}
    {\quad\quad\thecontentslabel\tabto{3.5cm}}
    {}
    {\hfill\contentspage}

% Change the Table of Contents, List of Tables ... etc. font size insited of big font
\renewcommand{\contentsname}{\normalsize{Table of Contents}}
\renewcommand{\listfigurename}{\normalsize{List of Figures}}
\renewcommand{\listtablename}{\normalsize{List of Tables}}
\renewcommand{\bibname}{\normalsize{Bibliography}}
\renewcommand{\indexname}{\normalsize{Index}}

% May 2009 added this to move page number up a bit
\addtolength{\voffset}{-2em}

% This is to format the chapter tags in this file
\usepackage{chngcntr}
\usepackage{lipsum}% just to generate text for the example

%% Page layout (customized to letter paper and Stevens requirements):
% if not using pdflatex to produce output, you may need to change to pagewidth and pageheight variables.
%\pagewidth 8.5in
%\pageheight 11in 
\pdfpagewidth 8.5in
\pdfpageheight 11in 
%
\setlength{\textheight}{8.5in} 
\setlength{\oddsidemargin}{0.5in}  
\setlength{\evensidemargin}{0.5in} 
\setlength{\textwidth}{6.0in}
\setlength{\topmargin}{0.in}    
\setlength{\headheight}{0.5in}
\setlength{\headwidth}{6.0in}
% change from .25in to .5 in May 2009 
\setlength{\headsep}{0.65in}
\setlength{\parindent}{12mm}

% For each chapter and section titles in the rest of the document the font formatting

% Chapter styles
\usepackage{sectsty}

\chapternumberfont{\normalsize} 
\chaptertitlefont{\normalsize}

\makeatletter
% No extra space between chapter number and chapter header lines:
\patchcmd{\@makechapterhead} {\vskip 20}{\vskip 0} {}{}
% Reduce extra space between chapter header and section header lines by 50%:
\patchcmd{\@makechapterhead} {\vskip 40}{\vskip 20}{}{}
\patchcmd{\@makeschapterhead}{\vskip 40}{\vskip 20}{}{} % for unnumbered chapters
\makeatother

% Sections styles
\sectionfont{\normalsize}

% Sub-sections styles
\subsectionfont{\normalsize}



%% Use the following commands, if desired, during production.
%% Comment them out for final version.
%\usepackage{layout} % defines the \layout command, see below
%\setlength{\hoffset}{-.75in} % creates a large right margin for notes and \showlabels

\pagestyle{fancy}
\fancyhf{}
% this prints a line under the header
\renewcommand{\headrulewidth}{0 pt}
%this prints a line under the footer
\renewcommand{\footrulewidth}{0 pt}
\fancyhead[RO]{}
\fancyhead[LO]{}
\fancyfoot[C]{}
\rhead{\thepage}

\fancypagestyle{plain}{%
\fancyhf{}
\rhead{\thepage}
}

%% Page layout (customized to letter paper and NYU requirements):
\setlength{\headheight}{20pt} 

%% Use the line below for official NYU version, which requires
%% double line spacing. For all other uses, this is unnecessary,
%% so the line can be commented out.
\onehalfspacing % requires package setspace, invoked above

%% Each of the following lines defines the \com command, which produces
%% a comment (notes for yourself, for instance) in the output file.
%% Example:    \com{this will appear as a comment in the output}
%% Choose (uncomment) only one of the three forms:
%\newcommand{\com}[1]{[/// {#1} ///]}       % between [/// and ///].
\newcommand{\com}[1]{\marginpar{\tiny #1}} % as (tiny) margin notes
%\newcommand{\com}[1]{}                     % suppress all comments.

%% Cross-referencing utilities. Use one or the other--whichever you prefer--
%% but comment out both lines for final version.
%\usepackage{showlabels}
%\usepackage{showkeys}
% \pagestyle{headings}

\begin{document}
%% Produces a test "layout" page, for "debugging" purposes only.
%% Comment out for final version.
%\layout % requires package layout (see above, on this same file)
%% Sets page numbering to "roman style" i, ii, iii, iv, etc:

%%%%%% Cover page %%%%%%%%%%%
%% Sets page numbering to "roman style" i, ii, iii, iv, etc:
\pagenumbering{roman}
\thispagestyle{empty}
\begin{center}
{
  {\thesistitle}
  \vspace{.15in}
  
    by
    
  \vspace{.15in}
  \thesisauthor
  
  \vspace{.15in}
  
 {A DISSERTATION}\\
  \vspace{.2in}
  \begin{spacing}{1}
    {Submitted to the Faculty of the Stevens Institute of Technology\\
    in partial fulfillment of the requirements for the degree of}
    \end{spacing}
  \vspace{.2in}
  
  {DOCTOR OF PHILOSOPHY}\\
  \vspace{1.0in}
  % for master thesis, change Chairman to Advisor
    \hfill 
    \begin{minipage}{80mm}
    \begin{spacing}{ }\noindent \rule{3.2in}{0.1mm}
        \thesisname, Candidate\\[3mm]
        \underline{ADVISORY COMMITTEE}\\[3mm]
        \noindent \rule{3.2in}{0.1mm}\\[-1.3mm]
        % for master thesis, change Chairman to Advisor
        {\thesischairadvisor}, Chairman  \hfill{Date}\\[2mm]
        {\noindent \rule{3.2in}{0.1mm}}\\[-1.3mm]
        {\committeenameA}        \hfill{Date}\\[2mm]
        {\noindent \rule{3.2in}{0.1mm}}\\[-1.3mm]
        {\committeenameB}        \hfill{Date}\\[2mm]
        {\noindent \rule{3.2in}{0.1mm}}\\[-1.3mm]
        {\committeenameC}        \hfill{Date}\\[2mm]
    \end{spacing}
  \end{minipage}
  \vfill
  
  {STEVENS INSTITUTE OF TECHNOLOGY\\
  \vspace{-0.05in}
  Castle Point on Hudson\\
  Hoboken, NJ 07030
  }
  % \vfill

  {\graddate}
}

\end{center}

\newpage



%%%%%%%%%%%%%% Microfilm / Publishing Page ProQuest %%%%%%%%%%%%%%%%%
% You can comment this section it is here to show you how it will appear when it submitted.
\thispagestyle{empty}
\begin{center}
ProQuest Number: XXXXXXXX

\vspace{.45in}

All rights reserved.

\vspace{.1in}

INFORMATION TO ALL USERS\\
The quality of this reproduction is dependent upon the quality of the copy submitted.
\vspace{.2in}

In the unlikely event hat  the author did not send a complete manuscript and there are missing pages, these will be  noted. Also ,if material had to be removed, a note will indicate the deletion.

\vspace{.1in}

\begin{figure}[H]
  \centering
  \includegraphics[width=200px]{misc/proquest-seeklogo.eps}
\end{figure}

\vspace{.1in}

ProQuest Number: XXXXXXXX

\vspace{.1in}

Published  by  ProQuest  LLC (\the\year).        Copyright of the Dissertation is held by the Author.

\vspace{.2in}

All rights reserved. This work is protected against  unauthorized copying under Title 17, United States Code Microform Edition {\textcopyright} ProQuest  LLC.

\vspace{.2in}

ProQuest LLC.\\
789 East Eisenhower Parkway\\
P.O. Box 1346\\
Ann Arbor, MI  48106-1346

\end{center}
\newpage

%%%%%% Copyrights page %%%%%%%%%%%
%
\setcounter{page}{2}
%% No numbering in the title page:
\thispagestyle{empty}
%
\begin{center}
{
  \vspace*{\fill}

  {\textcopyright} \graddate, \thesisname. All rights reserved.
}

\end{center}

\newpage
\doublespacing

%%%%%%%%%%%%%% Abstract %%%%%%%%%%%%%%%%%
\begin{center}
    {\thesistitle}\\ 
    {ABSTRACT}\\
    \vspace{.05in}
\end{center}
\addcontentsline{toc}{chapter}{Abstract}
%!TEX root = thesis.tex

%
\lipsum[2]
\vspace{0.4in}
\begin{flushleft}
Author: \thesisname \\
Advisor: \thesischairadvisor \\
Date: \thesisdate \\
Department: \thesisdepartment \\
Degree: \thesisdegree \\
\end{flushleft}
\newpage

%%%%%%%%%%%%%% Dedication Page %%%%%%%%%%%%%%%%%
%% Comment out the following lines if you do not want to dedicate it is optional
\chapter*{Dedication}
\addcontentsline{toc}{chapter}{Dedication}
\strut \vspace{2in}
\begin{center}
This dissertation is dedicated to my family, who supported me during my Ph.D. journey.
    \end{center}
    \vfill \strut
    \newpage
\newpage

%%%%%%%%%%%%%% Acknowledgements %%%%%%%%%%%%%%%%%
%% Comment out the following lines if you do not want to acknowledge
%% anyone's help...
\addcontentsline{toc}{chapter}{Acknowledgements}
%!TEX root = thesis.tex

%% Write your acknowledgements in this file. If you do not want to acknowledge anyone,
%% you can delete this file and comment out the corresponding part in the "thesis.tex"
%% file.
\noindent \textbf{Acknowledgments} \\[6mm]
The realization of this work was only possible due to the several people’s collaboration, which desires me to express my gratefulness.

\lipsum[2]

    \vfill \strut


\newpage

%%%% Table of Contents %%%%%%%%%%%%
\setcounter{tocdepth}{2} % To show a three level depth of sections

\tableofcontents

% \clearpage
% \pagestyle{headings}
\newpage

%%%%% List of Tables %%%%%%%%%%%%%
%% Comment out the following two lines if your thesis does not
%% contain any tables. The list of tables contains only
%% those tables included withing the "table" environment.
\listoftables
\newpage

%%%%% List of Figures %%%%%%%%%%%%%
%% Comment out the following two lines if your thesis does not
%% contain any figures. The list of figures contains only
%% those figures included withing the "figure" environment.
\listoffigures\addcontentsline{toc}{chapter}{\listfigurename}
\newpage

%%%%% Body of thesis starts %%%%%%%%%%%%
\pagenumbering{arabic} % switches page numbering to arabic: 1, 2, 3, etc.

%% Introduction. If your thesis has no introduction, or chapter 1 is
%% meant to be the introduction, then comment out the lines below.
%% \section*{Introduction}\addcontentsline{toc}{section}{Introduction}
%\input{intro}

%%If your thesis has different "Parts", use commands such as the following:

\chapter{Introduction}

\lipsum[2]

% At the end of each chapter
\newpage

\chapter{Two} \label{CH2}

\lipsum[2]

% At the end of each chapter
\newpage

\chapter{Three} \label{CH3}

\lipsum[2]

% At the end of each chapter
\newpage

\chapter{Four} \label{CH4}

\lipsum[2]

% At the end of each chapter
\newpage

\chapter{Five} \label{CH5}

\lipsum[2]

% At the end of each chapter
\newpage

\chapter{Six} \label{CH6}

\lipsum[2]

% At the end of each chapter
\newpage

%!TEX root = ../thesis.tex
\chapter{Conclusion and Future Works} \label{conc}

\lipsum[2]

% At the end of each chapter
\newpage
%%%%% Appendices start %%%%%%%%%%%%%%%%
%% Comment out the following line if your thesis has no appendix
\addcontentsline{toc}{chapter}{Appendices}
\noindent \textbf{Appendices}
\appendix
\addcontentsline{toc}{section}{Appendix A}  %% removed \\
\noindent \textbf{Appendix A}
\vspace{12pt}

\noindent Appendices at the end of a dissertation are optional, and depend on the content of the dissertation. There can be one or more appendicies, however they should retain the page numbering requirements for dissertations.  Any concerns about the formatting of an appendix should be brought to Doris Oliver, who can direct you how to format your appendix if you have questions.

\begin{center}
\begin{tabular}{|l|c|p{3.0in}|}
\hline
\multicolumn{3}{|c|}{Theoretical Dissertation Timeline}\\ \hline
Taskt & Time to Finish & Notes\\ \hline
Problem statement & 10 hours & Initially very upbeat.\\ \hline
Research & 3 days&Literature search to very previous studies.\\ \hline
Reformulation&4 hours&Presented and accepted by advisor\\ \hline
Research&20 days&Literature search to very previous  studies.\\ \hline
Experiments&14 days&Do some experiments and get results.\\ \hline
Format&1 day&Understand format guidelines for paper.\\ \hline
Write&years&Write the paper.\\ \hline
Revise&not too long&Proof read, etc.\\ \hline
Format&1-3 days&Verify correct report format is used.\\ \hline
See Library&1 hour&Meet with Doris to verify formatting.\\ \hline
Defend&1 day&Defend your research.\\ \hline
Revise&0 hours&It was perfect the first time.\\ \hline
Submit&1 day&Submit final dissertation to the library.\\ \hline
\end{tabular}
\end{center}
\newpage

\addcontentsline{toc}{section}{Appendix B}  %% removed \\
\noindent \textbf{Appendix B}
\vspace{12pt}

\noindent Another one! Here is more text to go.
\newpage



%% Note: If your thesis has more than one appendix, NYU requires a "list of
%% appendices" page before the body of the thesis. I don't provide the tools
%% to create that here, so you're on your own for that one... Sorry.
%\addcontentsline{toc}{section}{Appendix B}  %% removed \\
\noindent \textbf{Appendix B}
\vspace{12pt}

\noindent Another one! Here is more text to go.
\newpage



%%%% Input bibliography file %%%%%%%%%%%%%%%
% % NYU PhD thesis format. Original template created by José Koiller 2007--2008.
%% Updated by Anshul Vikram Pandey with new design guidelines. 2017--2018.
%%% Modified by Abdullah Khanfor for Stevens Institute of Technology PhD thesis format design guidelines 2019--2020. Link to the old template: https://github.com/khanfor/stevens-phd-thesis-dissertation-template
%%% Updated by Mofadal Alymani for Stevens Institute of Technology PhD thesis format design guidelines 2020--2021.

%% Use the first of the following lines during production to
%% easily spot "overfull boxes" in the output. Use the second
%% line for the final version.
% \documentclass[12pt,draft,letterpaper]{report}
% \documentclass[12pt,letterpaper]{report}
\documentclass[12pt]{report}
\usepackage{siunitx}
\usepackage{enumitem}
\usepackage{url}
\usepackage[breaklinks]{hyperref}
\def\UrlBreaks{\do\/\do-}
%% Replace the title, name, advisor name, graduation date and dedication below with
%% your own. Graduation months must be January, May or September.
\newcommand{\thesistitle}{Thesis Title}
\newcommand{\thesisauthor}{First Last}
\newcommand{\thesisadvisor}{Advisor}
\newcommand{\thesisyear}{2021}
\newcommand{\thesisname}{First Last}
\newcommand{\thesischairadvisor}{Dr. Advisor}    % this name prints on the title page as chairman and the abstract page as advisor
\newcommand{\committeenameA}{Dr. Member}
\newcommand{\committeenameB}{Dr. Member}
\newcommand{\committeenameC}{Dr. Member}
\newcommand{\thesisdepartment}{Department}
\newcommand{\thesisdate}{April 26, 2021}
\newcommand{\thesistype}{dissertation}
\newcommand{\thesisdegree}{Doctor of Philosophy}
\newcommand{\graddate}{\the\year} % like 2020, 2019, no month or day should be written
\newcommand{\thesissigline}[1]{%
  \leftline{\hbox to 2.5in{}\hrulefill}
  \endgraf
  \vspace*{-18pt}
  \leftline{\hbox to 2.53in{}{#1}}}

%% If you do not want a dedication, scroll down and comment out
%% the appropriate lines in this file.
\newcommand{\thesisdedication}{To all the Ph.D. pursuing brave souls}

%% The following makes chapters and sections, but not subsections,
%% appear in the TOC (table of contents). Increase to 2 or 3 to
%% make subsections or subsubsections appear, respectively. It seems
%% to be usual to use the "1" setting, however.
\setcounter{tocdepth}{1}

%% Sectional units up to subsubsections are numbered. To number
%% subsections, but not subsubsections, decrease this counter to 2.
\setcounter{secnumdepth}{3}

% Setting a gap between page number and text block

%% This inputs your auxiliary file with \usepackage's and \newcommand's:
%% It is assumed that that file is called "definitions.tex".
\input{definitions}


%% Use the following commands, if desired, during production.
%% Comment them out for final version.
%\usepackage{layout} % defines the \layout command, see below
%\setlength{\hoffset}{-.75in} % creates a large right margin for notes and \showlabels

\pagestyle{fancy}
\fancyhf{}
% this prints a line under the header
\renewcommand{\headrulewidth}{0 pt}
%this prints a line under the footer
\renewcommand{\footrulewidth}{0 pt}
\fancyhead[RO]{}
\fancyhead[LO]{}
\fancyfoot[C]{}
\rhead{\thepage}

\fancypagestyle{plain}{%
\fancyhf{}
\rhead{\thepage}
}

%% Page layout (customized to letter paper and NYU requirements):
\setlength{\headheight}{20pt} 

%% Use the line below for official NYU version, which requires
%% double line spacing. For all other uses, this is unnecessary,
%% so the line can be commented out.
\onehalfspacing % requires package setspace, invoked above

%% Each of the following lines defines the \com command, which produces
%% a comment (notes for yourself, for instance) in the output file.
%% Example:    \com{this will appear as a comment in the output}
%% Choose (uncomment) only one of the three forms:
%\newcommand{\com}[1]{[/// {#1} ///]}       % between [/// and ///].
\newcommand{\com}[1]{\marginpar{\tiny #1}} % as (tiny) margin notes
%\newcommand{\com}[1]{}                     % suppress all comments.

%% Cross-referencing utilities. Use one or the other--whichever you prefer--
%% but comment out both lines for final version.
%\usepackage{showlabels}
%\usepackage{showkeys}
% \pagestyle{headings}

\begin{document}
%% Produces a test "layout" page, for "debugging" purposes only.
%% Comment out for final version.
%\layout % requires package layout (see above, on this same file)
%% Sets page numbering to "roman style" i, ii, iii, iv, etc:

%%%%%% Cover page %%%%%%%%%%%
%% Sets page numbering to "roman style" i, ii, iii, iv, etc:
\pagenumbering{roman}
\thispagestyle{empty}
\begin{center}
{
  {\thesistitle}
  \vspace{.15in}
  
    by
    
  \vspace{.15in}
  \thesisauthor
  
  \vspace{.15in}
  
 {A DISSERTATION}\\
  \vspace{.2in}
  \begin{spacing}{1}
    {Submitted to the Faculty of the Stevens Institute of Technology\\
    in partial fulfillment of the requirements for the degree of}
    \end{spacing}
  \vspace{.2in}
  
  {DOCTOR OF PHILOSOPHY}\\
  \vspace{1.0in}
  % for master thesis, change Chairman to Advisor
    \hfill 
    \begin{minipage}{80mm}
    \begin{spacing}{ }\noindent \rule{3.2in}{0.1mm}
        \thesisname, Candidate\\[3mm]
        \underline{ADVISORY COMMITTEE}\\[3mm]
        \noindent \rule{3.2in}{0.1mm}\\[-1.3mm]
        % for master thesis, change Chairman to Advisor
        {\thesischairadvisor}, Chairman  \hfill{Date}\\[2mm]
        {\noindent \rule{3.2in}{0.1mm}}\\[-1.3mm]
        {\committeenameA}        \hfill{Date}\\[2mm]
        {\noindent \rule{3.2in}{0.1mm}}\\[-1.3mm]
        {\committeenameB}        \hfill{Date}\\[2mm]
        {\noindent \rule{3.2in}{0.1mm}}\\[-1.3mm]
        {\committeenameC}        \hfill{Date}\\[2mm]
    \end{spacing}
  \end{minipage}
  \vfill
  
  {STEVENS INSTITUTE OF TECHNOLOGY\\
  \vspace{-0.05in}
  Castle Point on Hudson\\
  Hoboken, NJ 07030
  }
  % \vfill

  {\graddate}
}

\end{center}

\newpage



%%%%%%%%%%%%%% Microfilm / Publishing Page ProQuest %%%%%%%%%%%%%%%%%
% You can comment this section it is here to show you how it will appear when it submitted.
\thispagestyle{empty}
\begin{center}
ProQuest Number: XXXXXXXX

\vspace{.45in}

All rights reserved.

\vspace{.1in}

INFORMATION TO ALL USERS\\
The quality of this reproduction is dependent upon the quality of the copy submitted.
\vspace{.2in}

In the unlikely event hat  the author did not send a complete manuscript and there are missing pages, these will be  noted. Also ,if material had to be removed, a note will indicate the deletion.

\vspace{.1in}

\begin{figure}[H]
  \centering
  \includegraphics[width=200px]{misc/proquest-seeklogo.eps}
\end{figure}

\vspace{.1in}

ProQuest Number: XXXXXXXX

\vspace{.1in}

Published  by  ProQuest  LLC (\the\year).        Copyright of the Dissertation is held by the Author.

\vspace{.2in}

All rights reserved. This work is protected against  unauthorized copying under Title 17, United States Code Microform Edition {\textcopyright} ProQuest  LLC.

\vspace{.2in}

ProQuest LLC.\\
789 East Eisenhower Parkway\\
P.O. Box 1346\\
Ann Arbor, MI  48106-1346

\end{center}
\newpage

%%%%%% Copyrights page %%%%%%%%%%%
%
\setcounter{page}{2}
%% No numbering in the title page:
\thispagestyle{empty}
%
\begin{center}
{
  \vspace*{\fill}

  {\textcopyright} \graddate, \thesisname. All rights reserved.
}

\end{center}

\newpage
\doublespacing

%%%%%%%%%%%%%% Abstract %%%%%%%%%%%%%%%%%
\begin{center}
    {\thesistitle}\\ 
    {ABSTRACT}\\
    \vspace{.05in}
\end{center}
\addcontentsline{toc}{chapter}{Abstract}
\input{abstract}
\vspace{0.4in}
\begin{flushleft}
Author: \thesisname \\
Advisor: \thesischairadvisor \\
Date: \thesisdate \\
Department: \thesisdepartment \\
Degree: \thesisdegree \\
\end{flushleft}
\newpage

%%%%%%%%%%%%%% Dedication Page %%%%%%%%%%%%%%%%%
%% Comment out the following lines if you do not want to dedicate it is optional
\chapter*{Dedication}
\addcontentsline{toc}{chapter}{Dedication}
\strut \vspace{2in}
\begin{center}
This dissertation is dedicated to my family, who supported me during my Ph.D. journey.
    \end{center}
    \vfill \strut
    \newpage
\newpage

%%%%%%%%%%%%%% Acknowledgements %%%%%%%%%%%%%%%%%
%% Comment out the following lines if you do not want to acknowledge
%% anyone's help...
\addcontentsline{toc}{chapter}{Acknowledgements}
\input{acknowledge}
\newpage

%%%% Table of Contents %%%%%%%%%%%%
\setcounter{tocdepth}{2} % To show a three level depth of sections

\tableofcontents

% \clearpage
% \pagestyle{headings}
\newpage

%%%%% List of Tables %%%%%%%%%%%%%
%% Comment out the following two lines if your thesis does not
%% contain any tables. The list of tables contains only
%% those tables included withing the "table" environment.
\listoftables
\newpage

%%%%% List of Figures %%%%%%%%%%%%%
%% Comment out the following two lines if your thesis does not
%% contain any figures. The list of figures contains only
%% those figures included withing the "figure" environment.
\listoffigures\addcontentsline{toc}{chapter}{\listfigurename}
\newpage

%%%%% Body of thesis starts %%%%%%%%%%%%
\pagenumbering{arabic} % switches page numbering to arabic: 1, 2, 3, etc.

%% Introduction. If your thesis has no introduction, or chapter 1 is
%% meant to be the introduction, then comment out the lines below.
%% \section*{Introduction}\addcontentsline{toc}{section}{Introduction}
%\input{intro}

%%If your thesis has different "Parts", use commands such as the following:

\input{introduction/introduction}

\input{two/two}

\input{three/three}

\input{four/four}

\input{five/five}

\input{six/six}

\input{conclusion/conclusion}
%%%%% Appendices start %%%%%%%%%%%%%%%%
%% Comment out the following line if your thesis has no appendix
\addcontentsline{toc}{chapter}{Appendices}
\noindent \textbf{Appendices}
\appendix
\input{appendix/app1}
\input{appendix/app2}

%% Note: If your thesis has more than one appendix, NYU requires a "list of
%% appendices" page before the body of the thesis. I don't provide the tools
%% to create that here, so you're on your own for that one... Sorry.
%\input{app2}

%%%% Input bibliography file %%%%%%%%%%%%%%%
% \input{thesis}
\bibliographystyle{abbrv}
% Remember to remove the repated artificates/papers from the different files
\bibliography{biblio,introduction/introduction,two/two,three/three,four/four,five/five,six/six,conclusion/conclusion} % add intelligently
\newpage

%%%%%%%%%%%%%% Vita %%%%%%%%%%%%%%%%%
\section*{Vita}
\addcontentsline{toc}{chapter}{Vita}
\input{vita}

\newpage

\end{document}

\bibliographystyle{abbrv}
% Remember to remove the repated artificates/papers from the different files
\bibliography{biblio,introduction/introduction,two/two,three/three,four/four,five/five,six/six,conclusion/conclusion} % add intelligently
\newpage

%%%%%%%%%%%%%% Vita %%%%%%%%%%%%%%%%%
\section*{Vita}
\addcontentsline{toc}{chapter}{Vita}
\begin{singlespace}
% Set your name here
\def\name{First Last}

% Custom section fonts
\sectionfont{\rmfamily\mdseries\normalsize}
\subsectionfont{\rmfamily\mdseries\itshape\normalsize}

% Other possible font commands include:
% \ttfamily for teletype,
% \sffamily for sans serif,
% \bfseries for bold,
% \scshape for small caps,
% \normalsize, \large, \Large, \LARGE sizes.

% Don't indent paragraphs.
\setlength\parindent{0em}

% Make lists without bullets
\renewenvironment{itemize}{
  \begin{list}{}{
    \setlength{\leftmargin}{1.5em}
  }
}{
  \end{list}
}

% Place name at left
{\LARGE \bf \name}

% Alternatively, print name centered and bold:
%\centerline{\huge \bf \name}

\vspace{0.25in}

\begin{minipage}{0.45\linewidth}
  \href{http://www.stevens.edu/}{Stevens Institute of Technology} \\
  Department \\
  Building \\
  Hoboken, NJ 07030
\end{minipage}
\begin{minipage}{0.45\linewidth}
  \begin{tabular}{ll}
    \\
    \\
    Email: & \href{mailto:xxxxxx@stevens.edu}{\tt xxxxxx@stevens.edu} \\
  \end{tabular}
\end{minipage}

\section*{\textbf{Place of Birth}}
\begin{itemize}
\item US.
\end{itemize}

\section*{\textbf{Date of Birth}}
\begin{itemize}
\item 1989.
\end{itemize}

\section*{\textbf{Education}}
\begin{itemize}
  \item B.S. .
  \item M.S. .
  \item Ph.D. .
\end{itemize}

\section*{\textbf{Employment}}
\begin{itemize}
\item .
\end{itemize}

\section*{\textbf{Publications}}
\begin{itemize}
\item .
\end{itemize}

\section*{\textbf{Honors \& Awards}}
\begin{itemize}
\item .
\end{itemize}

\end{singlespace}

\newpage

\end{document}

\bibliographystyle{abbrv}
% Remember to remove the repated artificates/papers from the different files
\bibliography{biblio,introduction/introduction,two/two,three/three,four/four,five/five,six/six,conclusion/conclusion} % add intelligently
\newpage

%%%%%%%%%%%%%% Vita %%%%%%%%%%%%%%%%%
\section*{Vita}
\addcontentsline{toc}{chapter}{Vita}
\begin{singlespace}
% Set your name here
\def\name{First Last}

% Custom section fonts
\sectionfont{\rmfamily\mdseries\normalsize}
\subsectionfont{\rmfamily\mdseries\itshape\normalsize}

% Other possible font commands include:
% \ttfamily for teletype,
% \sffamily for sans serif,
% \bfseries for bold,
% \scshape for small caps,
% \normalsize, \large, \Large, \LARGE sizes.

% Don't indent paragraphs.
\setlength\parindent{0em}

% Make lists without bullets
\renewenvironment{itemize}{
  \begin{list}{}{
    \setlength{\leftmargin}{1.5em}
  }
}{
  \end{list}
}

% Place name at left
{\LARGE \bf \name}

% Alternatively, print name centered and bold:
%\centerline{\huge \bf \name}

\vspace{0.25in}

\begin{minipage}{0.45\linewidth}
  \href{http://www.stevens.edu/}{Stevens Institute of Technology} \\
  Department \\
  Building \\
  Hoboken, NJ 07030
\end{minipage}
\begin{minipage}{0.45\linewidth}
  \begin{tabular}{ll}
    \\
    \\
    Email: & \href{mailto:xxxxxx@stevens.edu}{\tt xxxxxx@stevens.edu} \\
  \end{tabular}
\end{minipage}

\section*{\textbf{Place of Birth}}
\begin{itemize}
\item US.
\end{itemize}

\section*{\textbf{Date of Birth}}
\begin{itemize}
\item 1989.
\end{itemize}

\section*{\textbf{Education}}
\begin{itemize}
  \item B.S. .
  \item M.S. .
  \item Ph.D. .
\end{itemize}

\section*{\textbf{Employment}}
\begin{itemize}
\item .
\end{itemize}

\section*{\textbf{Publications}}
\begin{itemize}
\item .
\end{itemize}

\section*{\textbf{Honors \& Awards}}
\begin{itemize}
\item .
\end{itemize}

\end{singlespace}

\newpage

\end{document}

\bibliographystyle{abbrv}
% Remember to remove the repated artificates/papers from the different files
\bibliography{biblio,introduction/introduction,two/two,three/three,four/four,five/five,six/six,conclusion/conclusion} % add intelligently
\newpage

%%%%%%%%%%%%%% Vita %%%%%%%%%%%%%%%%%
\section*{Vita}
\addcontentsline{toc}{chapter}{Vita}
\begin{singlespace}
% Set your name here
\def\name{First Last}

% Custom section fonts
\sectionfont{\rmfamily\mdseries\normalsize}
\subsectionfont{\rmfamily\mdseries\itshape\normalsize}

% Other possible font commands include:
% \ttfamily for teletype,
% \sffamily for sans serif,
% \bfseries for bold,
% \scshape for small caps,
% \normalsize, \large, \Large, \LARGE sizes.

% Don't indent paragraphs.
\setlength\parindent{0em}

% Make lists without bullets
\renewenvironment{itemize}{
  \begin{list}{}{
    \setlength{\leftmargin}{1.5em}
  }
}{
  \end{list}
}

% Place name at left
{\LARGE \bf \name}

% Alternatively, print name centered and bold:
%\centerline{\huge \bf \name}

\vspace{0.25in}

\begin{minipage}{0.45\linewidth}
  \href{http://www.stevens.edu/}{Stevens Institute of Technology} \\
  Department \\
  Building \\
  Hoboken, NJ 07030
\end{minipage}
\begin{minipage}{0.45\linewidth}
  \begin{tabular}{ll}
    \\
    \\
    Email: & \href{mailto:xxxxxx@stevens.edu}{\tt xxxxxx@stevens.edu} \\
  \end{tabular}
\end{minipage}

\section*{\textbf{Place of Birth}}
\begin{itemize}
\item US.
\end{itemize}

\section*{\textbf{Date of Birth}}
\begin{itemize}
\item 1989.
\end{itemize}

\section*{\textbf{Education}}
\begin{itemize}
  \item B.S. .
  \item M.S. .
  \item Ph.D. .
\end{itemize}

\section*{\textbf{Employment}}
\begin{itemize}
\item .
\end{itemize}

\section*{\textbf{Publications}}
\begin{itemize}
\item .
\end{itemize}

\section*{\textbf{Honors \& Awards}}
\begin{itemize}
\item .
\end{itemize}

\end{singlespace}

\newpage

\end{document}
